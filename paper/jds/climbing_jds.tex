\documentclass[letterpaper, inpress]{jds} % use this for production
% \documentclass[a4paper, review]{jds}      % use this for review

%%%%%%%%%%%%%%%%%%%%%%%%%%%%%%%%%%%%%%%%%%%%%%%%%%%%%%%%%%%%%%%%%%%%%%
%% the following edits should be done by Journal typesetters
%%%%%%%%%%%%%%%%%%%%%%%%%%%%%%%%%%%%%%%%%%%%%%%%%%%%%%%%%%%%%%%%%%%%%%
\setcounter{page}{1}            % set the first page number
\jdsmonth{x}                 % month
\jdsyear{xxxx}                  % year
\jdsvolume{xx}                  % volume number
\jdsissue{xx}                   % issue number
\jdsdoi{xx.xxxx/xxxxxxxxx}      % doi
\jdsreceived{x, xxxx}       % optional: comment it out if no received date
\jdsaccepted{x, xxxx}         % optional: comment it out if no accepted date
%% manually set running header for a shorter list of authors if needed
% \shortauthors{A Author A, et al.}


%%%%%%%%%%%%%%%%%%%%%%%%%%%%%%%%%%%%%%%%%%%%%%%%%%%%%%%%%%%%%%%%%%%%%%
%% edits by authors are given below
%%%%%%%%%%%%%%%%%%%%%%%%%%%%%%%%%%%%%%%%%%%%%%%%%%%%%%%%%%%%%%%%%%%%%%
\usepackage{amsfonts,amsmath,amssymb,amsthm}
\usepackage{booktabs}

\usepackage{lipsum}

\title[An Examination of Sport Climbing Competition Format and Scoring System]{An Examination of Sport Climbing Competition Format and Scoring System}

%% corresponding authors can be labeled by either \thanks or \footnote
\author[1]{Quang Nguyen\footnote{Corresponding author. Email: nnguyen22@luc.edu}}
\author[2]{Hannah Butler}
\author[1]{Gregory J. Matthews}
\affil[1]{Department of Mathematics and Statistics, Loyola University Chicago, Chicago, IL, USA}
\affil[2]{Department of Statistics, Colorado State University, Fort Collins, CO, USA}


\begin{document}

\maketitle

\begin{abstract}
Sport climbing, which made its Olympic debut at the 2020 Summer Games, generally consists of three separate disciplines: speed climbing, bouldering, and lead climbing. However, the International Olympic Committee (IOC) only allowed one set of medals per gender for sport climbing. As a result, the governing body of sport climbing, rather than choosing only one of the three disciplines to include in the Olympics, decided to create a competition combining all three disciplines. In order to determine a winner, a combined scoring system was created using the product of the ranks across the three disciplines to determine an overall score for each climber. In this work, the rank-product scoring system of sport climbing is evaluated through simulation to investigate its general features, specifically, the advancement probabilities and scores for climbers given certain placements. Additionally, analyses of historical climbing contest results are presented and real examples of violations of the independence of irrelevant alternatives are illustrated. Finally, this work finds evidence that the current competition format is putting speed climbers at a disadvantage.\@
\end{abstract}

\begin{keywords} % alphabetical; excluding anything in the title already
sport climbing; rankings; sports statistics
\end{keywords}

\section{Introduction}%
\label{sec:intro}

This is a template for both the review version and the production
version of JDS articles. They corresponds to two options of document
class \texttt{jds}: \texttt{review} and \texttt{inpress}.


\lipsum[1-2]


\section{Equations}%
\label{sec:eq}


Weibull distribution has the virtue of being a mathematically tractable model
and is versatile in terms of its applications in reliability, life data
analysis, actuarial science and others. Apart from being a potential model in
survival analysis and reliability engineering, it has a vast domain of other
applications.

Equations are always parts of sentences, so they need to have
appropriate punctuations. To evaluate the distribution of a normal
variable, one use
\begin{equation}
  \label{eq:cdf}
  \Pr(Z \le t) = \Phi\left(\frac{Z - \mu}{\sigma} \right),
\end{equation}
where $Z$ follows a $N(\mu, \sigma^2)$ distribution.
Equations can be referenced by \texttt{$\backslash$eqref}.
When $\mu = 0$ and $\sigma = 1$, the $Z$ in Equation~\eqref{eq:cdf}
becomes a standard normal variable.


Multiline equations can be presented with the \texttt{align}
environment. For example,
\begin{align*}
  g_{\mu}(\phi) = 0,\\
  g_{\mu}(X) = 1.
\end{align*}


An equation that is not referenced should not be labeled. The starred
version of the \texttt{equation} and \texttt{align} are for this purpose.

\section{Tables}%
\label{sec:tabs}

We recommend \LaTeX\ package \texttt{booktabs} for professional
looking tables. Its toprule and bottomrule are thicker than midrule.

A professional table contains no vertical lines.

Table~\ref{tab:realdata} is an illustration.
\begin{table}[tbp]
  \caption{Analysis results for real data. Point
    estimates (EST) from both two-piece method and marginal method are
    reported. Standard error of point estimates are evaluated by
    parametric bootstrap (SE).}%
  \label{tab:realdata}
  \centering
  \begin{tabular}{llrrrr}
    \toprule
    \multicolumn{1}{l}{Season} & \multicolumn{1}{l}{Parameter}
    & \multicolumn{2}{c}{Two-piece method} & \multicolumn{2}{c}{Marginal method} \\
    \cmidrule(r){3-4} \cmidrule(r){5-6}
                               & & EST & SE & EST & SE \\
    \midrule
    Summer & $\lambda_1$ & 2.841 & 0.459 & 1.090 & 0.280 \\
   & $\lambda_0$ & 0.179 & 0.014 & 0.158 & 0.015 \\
   & $\sigma$ & 1.335 & 0.106 & 0.999 & 0.104 \\
   & $\sigma_\epsilon (\times 10^{-2})$ & 1.854 & 0.087 & 1.879 & 0.078 \\
  Winter & $\lambda_1$ & 6.225 & 0.825 & 4.720 & 0.711 \\
   & $\lambda_0$ & 0.118 & 0.010 & 0.114 & 0.009 \\
   & $\sigma$ & 1.506 & 0.095 & 1.454 & 0.089 \\
   & $\sigma_\epsilon (\times 10^{-2})$ & 0.908 & 0.036 & 0.934 & 0.043 \\
    \bottomrule
  \end{tabular}
\end{table}




\section{Figures}%
\label{sec:figs}

Vector graphics do not lose clarity when being scaled. Make your
figure in pdf format when you first generate it and keep in mind its
sizes in the article to avoid over-scaling. Do not simply convert a
jpeg or png image to a pdf.

\section{Code}%
\label{sec:code}

The document class \texttt{jds} provides several commands to decorate
\begin{itemize}
\item inline code, such as \code{print("Hello world!")};
\item programming language, such as \proglang{R}, \proglang{Python}, and
  \proglang{C++};
\item software package, such as \pkg{stats}, \pkg{utils}.
\end{itemize}


\section{Guide for Authors}%
\label{sec:guide-authors}

The following requirements must be followed as closely as possible. A
technically acceptable manuscript that fails to follow these requirements may be
returned for retyping, leading to delay in publication.We only accept
submissions in PDF format. The Latex file must be provided after the manuscript
is accepted.

\subsection{Submission of Papers}

Submission of a manuscript must be the original work of the author(s) and have
not been published elsewhere or under consideration for another publication, or
a substantially similar form in any language.

Authors are encouraged to recommend three to five individuals (including their
research fields, e-mail, phone numbers and addresses) who are qualified to serve
as referees for their paper.


% \subsection{Manuscript Preparation}

% All manuscripts should be written in English. Letters are generally no more than
% three journal pages. The supporting organization and the grant number should be
% given at the end of the manuscript. For more details of submission format,
% please visit the website:

% \begin{itemize}
% \item Title: The title of the paper should be concise but informative.
% \item Author name(s): A list of all authors, as well as corresponding addresses,
%   should be provided on the title page. Authors’ names should be given in a
%   consistent form on all publications to facilitate indexing. It will be better
%   if the fax number(s), e-mail address(es), and telephone number(s) are all
%   provided.
% \item Abstract: The abstract should be no longer than 100 words. It should be
%   informative, without descriptive words or citations, and contain the major
%   conclusions and quantitative results or other significant items in the
%   paper. Together with the title, the abstract must be adequate as an index to
%   all the subjects treated in the paper, and will be used as a base for
%   indexing.
% \item Main body of the paper: The body of the paper should include all the
%   information of the research, but no subtitles.
% \item Formulas: Formulas should be punctuated and aligned to bring out their
%   structure, and numbered consecutively in round brackets on the right-hand side
%   of the page.
% \item Notation: Notation must be legible, clear, compact, and consistent with
%   standard usage. All unusual symbols whose identity may not be obvious,
%   including subscript or superscript, must be made comprehensible. Physical and
%   mathematical variables should be in italic, vectors in boldface. Units,
%   abbreviations and special functions should be upright. Please add notes to
%   explain any other special symbols.
% \item Figures: Figures should be original laser prints with high contrast,
%   suitable for immediate reproduction, typed on separate sheets and identified
%   by its number. They will normally be reduced to one column width (6--8 cm). In
%   the figures, the main lines should be about 0.3 mm in width, and the assistant
%   lines 0.15 mm. Notations in the figures should be distinct and consistent with
%   the same ones in the text, and their font size will be 7--9 pt. The positions
%   of figures should be marked in the text by boxes of a suitable size. Each
%   figure should have its own caption. For photographs, the original photos must
%   be supplied with good contrast and clearly distinguishable details.
% \item Tables: Tables, numbered in order of appearance, should be appended on
%   separate sheets and identified with appropriate titles.The table title, which
%   should be brief, goes above the table. A detailed description of its contents
%   or table footnotes should be given directly below the body of the table.
% \item References: References must be published work, and numbered consecutively
%   in order of their first citation. References should be listed individually at
%   the end of the text and indicated in the text with a superscript number in
%   square brackets. All of the references’ authors, as well as the titles of the
%   referenced articles, should be given.
% \item Proofs: Authors will receive a letter informing them whether the
%   manuscript is accepted or rejected in two months. Authors should return their
%   revisions to the Manuscript Office within one month on receipt. When an
%   article has been amended in compliance with the comments of referee(s), an
%   electronic file of the final version should be sent with the revised
%   manuscript. Proofs will be sent to the authors and should be returned
%   preferably within 48 hours to avoid publication delays.
% \end{itemize}


\section{A Placeholder Section}%
\label{sec:placeholder}

\lipsum[4-7]

\section{Citing References}%
\label{sec:citations}

The citations are in the author-year format with the
\texttt{chicago} bibstyle.

A bibfile contains all the citations in
bibtex format is prepared (see \texttt{JDSbib.bib}). Some characters
in the title of the references needs to be protected so that the style
file does not alter it. For example, in the bibtex entry for
\citet{Pozd:etal:2017}, the ``B'' in ``Brownian'' and the ``E'' in
``estimation''  following the colon are protected.
A book reference \citep[e.g.,][]{Kotz2001} should have an address field.


Citations can be in either text or parenthesis format style with
\texttt{$\backslash$citet} or \texttt{$\backslash$citep},
respectively. For example, \citet{KoenkerBassett1978} is a seminal
work on quantile regression; The Laplace distribution has applications
in many fields \citep{Kotz2001}.


\section{Discussion}

\lipsum[1-3]

\bibliographystyle{jds}
\bibliography{JDSbib}


\end{document}
